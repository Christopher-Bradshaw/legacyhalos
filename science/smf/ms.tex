\documentclass[twocolumn]{aastex62}
%\documentclass[modern]{aastex62}
\usepackage{amsmath}

\newcommand\latex{La\TeX}
\newcommand\redmapper{redMaPPer}
\newcommand\isedfit{\texttt{iSEDfit}}
\newcommand\LS{\textit{Legacy Surveys}}
\newcommand\sersic{S\'{e}rsic}
\newcommand\mstar{\ensuremath{\mathcal{M}_{*}}}
\newcommand\msun{\ensuremath{\mathcal{M}_{\odot}}}
\newcommand\Mcenscatter{\ensuremath{\sigma_{\log\,\mathcal{M}_{*}}}}

\graphicspath{{./}{figures/}}

%\received{January 1, 2018}
%\revised{January 7, 2018}
%\accepted{\today}
\submitjournal{ApJ}

\shorttitle{Sample article}
\shortauthors{Schwarz et al.}

\begin{document}

\title{{\em Legacyhalos} I: The Scatter in the Stellar Masses of Central Galaxies in Massive Dark Matter Halos}

\correspondingauthor{John Moustakas}
\email{jmoustakas@siena.edu}

\author[0000-0002-2733-4559]{John Moustakas}
\affil{Department of Physics \& Astronomy, Siena College, 515 Loudon Road, Loudonville, NY 12110, USA}

\author{Dustin Lang}
\affiliation{Department of Astronomy \& Astrophysics, Dunlap Institute, University of Toronto, Toronto, ON, M5S 3H4, Canada}

\author{et al.}
\affiliation{various}

\begin{abstract}
Measurements of the scatter in the stellar masses of central galaxies, \Mcenscatter, place strong constraints on galaxy formation models in the context of the hierarchical assembly of dark matter halos.  We combine new deep optical and mid-infrared photometry from the \LS{} with state-of-the-art stellar population synthesis models to measure the integrated stellar masses of $\approx200,000$ central galaxies at $0.05<z<0.5$ in $\approx10^{13}-10^{15}~\msun$ dark matter halos.  We find a mean intrinsic scatter of $\Mcenscatter=XXX\pm XXX$ for central galaxies with stellar masses $\mstar=10^{XX}-10^{XX}~\msun$ with {\bf little/weak/no} dependence on halo mass or redshift.  {\bf Redshift/halo dependence; non-Gaussianity; Mstar/Mhalo ratio; constraints on models.}
\end{abstract}

\keywords{not sure yet --- not sure}

\section{Introduction}\label{sec:intro}

Dark matter halos are the cradles within which galaxies form and evolve.  Central galaxies occupy a privileged position at, or near the center of dark matter halo.  

The scatter in the stellar masses of central galaxies is a key diagnostic of galaxy formation.  (1) Test of hierarchical assembly; (2) star formation efficiency, feedback, and galaxy quenching; (3) how baryons are distributed in dark matter halos.  Introduce the stellar mass function and the scatter in the stellar masses of central galaxies.

Various observational techniques are sensitive to the scatter in the stellar mass in dark matter halos (see \citealt{wechsler18a} for a recent, thorough review), although they each have their strengths and weaknesses:  (1) clustering (tinker17a, zu15a); (2) satellite kinematics; (3) galaxy groups and clusters.  


\vspace{1cm}
{\bf Other related refs (possibly for Paper II or general reading:
zhang2016, giallongo2013, dolag2010, cui14;   mccarthy17 -- compendium of hot gas fractions as a function of halo mass -- use this for a baryon fraction paper.  Romeel's latest simba simulations has a measurement of the mass fraction vs halo mass that could be overlaid;  Look at Peter's The Universe Machine or Fakhuri+09 for DM accretion histories;  Mike Hudson's talk from Kingston is has a nice model of how f* versus M* evolves with redshift;  Tight link between R200 and Reff: see Kravtson+13.}


Large sample with deeper optical and mid-IR photometry.  Plus in the future: DESI spectroscopy.


\section{Data \& Sample Selection}\label{sec:sample}

The \LS{} are observing $\approx14,000$~deg$^{2}$ of the extragalactic sky in three optical bandpasses, $g$, $r$, and $z$ approximately two magnitudes deeper than the Sloan Digital Sky Survey (SDSS; {\bf ref}).  In addition, {\bf latest WISE stacks} from \citet{meisner18a}.  See \citet{dey18a} for an overview of the surveys.  Should mention DESI \citep{desi-collaboration16a, desi-collaboration16b}.  Talk about depth; use the galaxy depth but it really doesn't matter.  Two-pass coverage of the sky to achieve full depth (and on average three passes to fill the chip gaps). 

We begin with the \redmapper{} v6.3.1 catalog\footnote{\url{http://risa.stanford.edu/redmapper}}, which contains XXX central galaxies.  Need to talk about the redshifts, photoz precision, and $P_{cen}$ (on which we do not cut).  The sample is restricted to the redshift range $z=0.05-0.6$.

\begin{figure*}[!ht]
\centering\includegraphics[width=0.8\textwidth]{montage}
\caption{Montage\label{fig:montage}}
\end{figure*}

We then cross-match (using a 1\arcsec{} search radius) with the \LS{} \emph{sweep} files and keep the nearest match, finding XXX centrals in the \LS{} footprint.  Next, we remove galaxies with shallower-than final imaging, leaving a final sample of 229,493 galaxies.

Figure~\ref{fig:montage} shows a $grz$ false-color montage of 100 randomly selected central galaxies in our sample, and Figure~\ref{fig:radec} shows the spatial distribution of the galaxies in our sample in a XXX projection.

\begin{figure}
\centering\includegraphics[width=0.5\textwidth]{radec}
\caption{Use a proper basemap, quantify the area, plot the surface density of sources.\label{fig:radec}}
\end{figure}

In Figure~\ref{fig:hist} we show some histograms (maybe add apparent magnitude?).  We also need richness vs redshift.

\begin{figure*}
\centering\includegraphics[width=0.9\textwidth]{zhist-lambdahist}
\caption{(\emph{Left}) Redshift and (\emph{right}) and cluster richness distribution for the sample. \label{fig:hist}}
\end{figure*}

Halo mass-richness relation: simet16a, mcclintlok18a (weak lensing), many others.  

\section{Spectral Energy Distribution Modeling}\label{sec:mass}

We use \isedfit{} to infer the stellar masses from the broadband photometry, given the redshift from \redmapper.  We assume minimum photometric uncertainties of $2\%$ in all five $grzW1W2$ bandpasses.  

Priors:  \citet{chabrier03a} initial mass function, delayed SFHs, metallicity, dust.

\section{Results}\label{sec:results}

\subsection{Stellar Mass Function}\label{sec:smf}

Figures: (1) histogram distribution of stellar masses (all galaxies and then split by richness and redshift).  Should compare different stellar mass outputs, different codes, etc.

\subsection{Scatter in the Stellar Masses of Central Galaxies}\label{sec:scatter} 

Plot the scatter (and also maybe the skew and kurtosis) as a function of richness / halo mass and redshift.

\section{Discussion}\label{sec:discussion}

Need a nice comparison of previous measurements and with theoretical "predictions".  Would be nice to make my own version of Fig 8 from \citet{wechsler18a}.  

\begin{itemize}
\item{Size-mass relation.  See genel17a from Illustris-TNG}
\item{}
\end{itemize}

\section{Summary}\label{sec:conclusions}

We have done lots of great stuff.

\clearpage

\acknowledgements

JM gratefully acknowledges partial funding support for this work by the National Science Foundation (NSF) grant AST-1616414.

The Legacy Surveys consist of three individual and complementary projects: the Dark Energy Camera Legacy Survey (DECaLS; NOAO Proposal ID \# 2014B-0404; PIs: David Schlegel and Arjun Dey), the Beijing-Arizona Sky Survey (BASS; NOAO Proposal ID \# 2015A-0801; PIs: Zhou Xu and Xiaohui Fan), and the Mayall $z$-band  Legacy Survey (MzLS; NOAO Proposal ID \# 2016A-0453; PI: Arjun Dey). DECaLS, BASS and MzLS together include data obtained, respectively, at the Blanco telescope, Cerro Tololo Inter-American Observatory, National Optical Astronomy Observatory (NOAO); the Bok telescope, Steward Observatory, University of Arizona; and the Mayall telescope, Kitt Peak National Observatory, NOAO. The Legacy Surveys project is honored to be permitted to conduct astronomical research on Iolkam Du'ag (Kitt Peak), a mountain with particular significance to the Tohono O'odham Nation.

NOAO is operated by the Association of Universities for Research in Astronomy (AURA) under a cooperative agreement with the National Science Foundation.

This project used data obtained with the Dark Energy Camera (DECam), which was constructed by the Dark Energy Survey (DES) collaboration. Funding for the DES Projects has been provided by the U.S. Department of Energy, the U.S. National Science Foundation, the Ministry of Science and Education of Spain, the Science and Technology Facilities Council of the United Kingdom, the Higher Education Funding Council for England, the National Center for Supercomputing Applications at the University of Illinois at Urbana-Champaign, the Kavli Institute of Cosmological Physics at the University of Chicago, Center for Cosmology and Astro-Particle Physics at the Ohio State University, the Mitchell Institute for Fundamental Physics and Astronomy at Texas A\&M University, Financiadora de Estudos e Projetos, Fundacao Carlos Chagas Filho de Amparo, Financiadora de Estudos e Projetos, Fundacao Carlos Chagas Filho de Amparo a Pesquisa do Estado do Rio de Janeiro, Conselho Nacional de Desenvolvimento Cientifico e Tecnologico and the Ministerio da Ciencia, Tecnologia e Inovacao, the Deutsche Forschungsgemeinschaft and the Collaborating Institutions in the Dark Energy Survey. The Collaborating Institutions are Argonne National Laboratory, the University of California at Santa Cruz, the University of Cambridge, Centro de Investigaciones Energeticas, Medioambientales y Tecnologicas-Madrid, the University of Chicago, University College London, the DES-Brazil Consortium, the University of Edinburgh, the Eidgenossische Technische Hochschule (ETH) Zurich, Fermi National Accelerator Laboratory, the University of Illinois at Urbana-Champaign, the Institut de Ciencies de l'Espai (IEEC/CSIC), the Institut de Fisica d'Altes Energies, Lawrence Berkeley National Laboratory, the Ludwig-Maximilians Universitat Munchen and the associated Excellence Cluster Universe, the University of Michigan, the National Optical Astronomy Observatory, the University of Nottingham, the Ohio State University, the University of Pennsylvania, the University of Portsmouth, SLAC National Accelerator Laboratory, Stanford University, the University of Sussex, and Texas A\&M University.

BASS is a key project of the Telescope Access Program (TAP), which has been funded by the National Astronomical Observatories of China, the Chinese Academy of Sciences (the Strategic Priority Research Program "The Emergence of Cosmological Structures" Grant \# XDB09000000), and the Special Fund for Astronomy from the Ministry of Finance. The BASS is also supported by the External Cooperation Program of Chinese Academy of Sciences (Grant \# 114A11KYSB20160057), and Chinese National Natural Science Foundation (Grant \# 11433005).

The Legacy Survey team makes use of data products from the Near-Earth Object Wide-field Infrared Survey Explorer (NEOWISE), which is a project of the Jet Propulsion Laboratory/California Institute of Technology. NEOWISE is funded by the National Aeronautics and Space Administration.

The Legacy Surveys imaging of the DESI footprint is supported by the Director, Office of Science, Office of High Energy Physics of the U.S. Department of Energy under Contract No. DE-AC02-05CH1123, by the National Energy Research Scientific Computing Center, a DOE Office of Science User Facility under the same contract; and by the U.S. National Science Foundation, Division of Astronomical Sciences under Contract No. AST-0950945 to NOAO.

\appendix

\section{Updated Photometry for \redmapper{} Central Galaxies}

See this notebook.\footnote{\url{https://github.com/moustakas/legacyhalos/blob/master/nb/redmapper-casjobs.ipynb}}

The goal of this notebook is to document how we retrieve the complete set of updated (DR14) SDSS ugriz and unWISE W1-W4 (forced) photometry for the parent sample of redMaPPer central galaxies.

%\acknowledgments
Thank nice people.

\bibliographystyle{aasjournal}
\bibliography{ioannis}

\end{document}
