\section{Conclusions \& Next Steps}\label{sec:summary}

This \tnote{} reports on our ongoing effort to develop a modified version of the
{\tt legacypipe} photometric pipeline which is optimized for large galaxies,
with the ultimate goal of producing a Legacy Survey Large Galaxy Atlas (LSLGA).
In addition to enabling detailing multi-wavelength analyses of large, spatially
well-resolved galaxies, this effort will also significantly improve DESI target
selection for the Bright Galaxy Survey and in pointings containing large
galaxies.

In addition to addressing the outstanding issues discussed above, in the near
future we intend to work on the following additional threads, in no particular
order of importance:
\begin{itemize}
\item{We will revisit the parent galaxy sample selection to ensure that low
  surface brightness galaxies are not being excluded (e.g., by incorporating
  H~{\sc i}-selected galaxies).}
\item{We will run the large-galaxy pipeline on the full sample of $10,654$
  galaxies, in order to identify the most common problems and failure modes.}
\item{We will build a web-based interface on the {\tt legacysurvey.org}
  web-server (work that we have already begun), in order to make it easy to
  inspect the data and model outputs for the large-galaxy sample.}
\item{Using existing code, we will use image simulations---whereby we insert
  simulated large galaxies into the CCD-level data---to test the performance of
  the large-galaxy pipeline under different initial conditions and in different
  regimes.}
\item{We will incorporate more sophisticated galaxy models, including
  multi-component S\'{e}rsic profiles, into the library of possible
  two-dimensional galaxy models.}
\item{We will investigate ways of accounting for second-order variations in the
  galaxy models, such as radial color gradients or azimuthal asymmetries.}
\item{And finally we will engage with the Data Systems and Target Selection
  Working Group to incorporate the catalogs constructed using the large-galaxy
  pipeline into DESI target selection and fiber assignment.}
\end{itemize}

%Planar sky.

%We zoom in on the object of interest and define a custom ``brick'' centered on
%that galaxy -- no edge issues to worry about!

%\begin{figure}
%\centering
%\includegraphics[width=0.3\textwidth]{figures/ugc04203-image-custom-annot.jpg}
%\includegraphics[width=0.3\textwidth]{figures/ugc04203-model-custom-annot.jpg}
%\includegraphics[width=0.3\textwidth]{figures/ugc04203-resid-custom-annot.jpg}
%\caption{Modeling of UGC04203 using the customized, large-galaxy optimized
%  version of the pipeline.  See Section~\ref{sec:largepipeline} for details.
%\label{fig:custom}}  
%\end{figure}
