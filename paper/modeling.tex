\section{Modeling Large Galaxies}\label{sec:modeling}

To highlight our progress modeling the sample of large galaxies, we focus on one
test case, UGC04203 (also known as the Phoenix galaxy), a face-on Sa galaxy with
an angular diameter of $0.85$~arcmin.  In Section~\ref{sec:production} we
summarize the performance of the production (DR2) version of the {\tt
  legacypipe} pipeline, in Section~\ref{sec:sky} we discuss masking and
sky-subtraction around large galaxies, and in Section~\ref{sec:largepipeline} we
show preliminary results from a version of the pipeline which is customized for
large galaxies.

%\begin{figure}%[!ht]
%\centering
%\includegraphics[width=0.3\textwidth]{figures/ugc04203-image-runbrick-annot.jpg}
%\includegraphics[width=0.3\textwidth]{figures/ugc04203-model-runbrick-annot.jpg}
%\includegraphics[width=0.3\textwidth]{figures/ugc04203-resid-runbrick-annot.jpg}
%\caption{Modeling of UGC04203 using the production (DR2) version of the
%  {\tt legacypipe} pipeline. 
%%  See Section~\ref{sec:production} for details.
%  \label{fig:runbrick}} 
%\end{figure}

\subsection{Production Pipeline}\label{sec:production}

In Figure~\ref{fig:runbrick} we illustrate the performance of the DR2 version of
the {\tt legacypipe} pipeline.  The left, middle, and right panels show the
data, model, and residual (data minus model) color mosaics.  The small circles
in each panel identify detected sources, where the color of the circle
corresponds to the final morphological classification of the object: {\tt
  PSF}=white, {\tt SIMP}=red, {\tt EXP}=orange, {\tt DEV}=cyan, and {\tt
  COMP}=yellow.\footnote{See http://legacysurvey.org/dr2/catalogs for a
  description of these classifications.}  Sources are identified as $5 \sigma$
peaks above the rms noise of the image after convolving with a Gaussian kernel
whose full-width at half-maximum (FWHM) is given by the median seeing of the
data. 

Overall, we find that the DR2 pipeline is doing a terrible job of modeling the
large, central galaxy.  Although UGC04203 has been successfully classified as a
{\tt COMP} (linear combination of an exponential plus de~Vaucouleurs
surface-brightness profile), its center has been significantly over-subtracted
and the outer envelope has been under-subtracted.  Moreover, the central panel
(model) shows clear aliasing in Fourier space.  Finally, we find that several
significant sources have not been identified, including two objects in the outer
envelope of UGC04203 (lower-right quadrant) and one source in the bottom-left
part of the footprint.  Failing to detect these sources creates several
significant issues when the pipeline attempts to optimize the parameters
(shapes, sizes, and surface-brightness profiles) of the sources it \emph{does}
detect.

\subsection{Improved Masking and Sky Subtraction}\label{sec:sky}

Unlike the typical, ``small'' stars and galaxies which appear in Legacy Survey
imaging, large galaxies have extended outer envelopes which may include a
significant amount of light.  Consequently, both object masking and
sky-subtraction are critically important.  Masking and sky-subtraction, of
course, must occur at the CCD-level data, since the {\tt Tractor} operates from
the unremapped, unstacked, pixel-level data.

To illustrate the available multi-band imaging for UGC04203, in
Figure~\ref{fig:ccdpos} we plot the positions of the $g$-band (left panel),
$r$-band (middle panel), and $z$-band (right panel) exposures, where each
individual CCD has been color-coded as indicated in the legend.  For reference,
the size of the dashed square is five times the angular diameter of the galaxy,
and the footprint covered by the square corresponds to the sky region shown in
Figure~\ref{fig:runbrick}.  Finally, the small circle at the center of each
panel in Figure~\ref{fig:ccdpos} indicates the angular diameter of UGC04203.

%\begin{figure}%[!ht]
%\centering
%\includegraphics[width=0.9\textwidth]{figures/qa-ugc04203-ccdpos.png}
%\caption{Positions of the CCDs for the available (left to right) $g$-, $r$-, and
%  $z$-band imaging of UGC04203.  The dashed square is five times the angular
%  diameter of the galaxy and the small circle at the center of each panel
%  indicates the angular diameter of UGC04203.  \label{fig:ccdpos}}
%\end{figure}

In Figure~\ref{fig:qaccd} we focus on one of these CCDs---{\tt ccd04}, an
$r$-band exposure with UGC04203 positioned reasonably close to the center of the
field---in more detail.  In this figure we show the image (including the
sky), the object mask constructed by the DR2 pipeline (one and zero indicate
masked and unmasked pixels, respectively), the object mask generated by the
large-galaxy pipeline, the DR2 model of the sky, and finally the model of the
sky based on the customized large-galaxy pipeline.  The white circle in each
panel shows the position and angular extent of UGC04203.

Figure~\ref{fig:qaccd} shows two key ideas.  First, the large-galaxy pipeline is
much more aggressive at masking pixels containing astrophysical sources,
especially the outer isophotes of UGC04203.  And second, the production version
of the pipeline clearly over-subtracts the sky in and around UGC04203, whereas
the large-galaxy pipeline (currently) subtracts a uniform sky background from
the data.  Other diagnostic plots (not shown here) indicate that the
low-resolution spline sky-subtraction implemented in the DR2 version of {\tt
  legacypipe} is overly aggressive (flexible) around large galaxies, and can
lead to the kind of over-subtraction systematics shown in Figure~\ref{fig:qaccd}
(fourth panel from the left).

%The units of the images are nanomaggies.

%\begin{figure}%[!ht]
%\centering
%\includegraphics[width=0.5\textwidth]{figures/qa-ugc04203-blobs.png}
%\caption{Blobs.  \label{fig:blobs}}
%\end{figure}

%\begin{figure}[!ht]
%\centering
%\includegraphics[width=1.0\textwidth]{figures/qa-ugc04203-ccd04-2d.png}
%\caption{CCD-level diagnostic plots for a single $r$-band exposure of UGC04203
%  (marked with a cirle).  From left to right we show the image, the object mask
%  constructed by the DR2 pipeline, the object mask generated by the large-galaxy
%  pipeline, the DR2 model of the sky, and finally the model of the sky based on
%  the customized large-galaxy pipeline. \label{fig:qaccd}}
%\end{figure}

\subsection{Large-Galaxy Pipeline}\label{sec:largepipeline}

Figure~\ref{fig:custom} shows the final result of running the customized
large-galaxy pipeline on UGC04203.  As in Figure~\ref{fig:runbrick} (see also
Section~\ref{sec:production}), we show---from left to right---a color montage of
the data, the model, and the residuals, and we show the detected sources and
their classification using colored circles.  Overall, we find a significant
improvement in the two-dimensional model of UGC04203, which as before is
classified as a composite galaxy.  In particular, the residual image shows
astrophysically interesting features, including spiral structure and low-level
shells at large galactocentric radius, the latter of which is generally cited as
evidence of a merger-driven origin for spheroidal and bulge-dominated galaxies. 

One outstanding issue shown in Figure~\ref{fig:custom}, however, is the source
in the lower-left quadrant which fails to be detected.  To compensate for this
extraneous flux, the pipeline chooses a thin, almost needle-shaped galaxy
profile for the adjacent source.  Although this issue does not specifically
affect our model of UGC04203, it is a recurring issue which we hope to address
in a future version of the {\tt legacypipe} pipeline.

