\section{Introduction}\label{sec:intro}

The Dark Energy Spectroscopic Instrument (DESI) will select spectroscopic
targets using data from three precursor ground-based optical imaging
surveys---DECaLS (DECam Legacy Survey), MzLS (Mayall $z$-band Legacy Survey),
and BASS (Beijing-Arizona Sky Survey).\footnote{\url{legacysurvey.org}} The
positions, shapes, sizes, colors, and other observable properties of the
galaxies and stars in these datasets are being measured by the {\tt
  legacypipe}\footnote{\url{https://github.com/legacysurvey/legacypipe}}
photometric pipeline, which uses the {\tt
  Tractor}\footnote{\url{http://thetractor.org}} to build a probabilistically
justified model of each source.

%and in particular target selection for the 

However, large galaxies---defined here to be galaxies with large (angular) sizes
projected on the sky, typically larger than $5-10$~arcsec---are not being
modeled properly by the production version of the pipeline, which will have
significant implications for DESI target selection, especially the Bright Galaxy
Survey (BGS).  A closely related issue is that each DESI pointing of 5000 fibers
will contain at least one galaxy $30$~arcsec or larger, so by analyzing large
galaxies we will be able to identify and select the desired fiber positions in
fields containing large galaxies (e.g., along the major axis or on the most
prominent star-forming regions).  Finally, large galaxies should be studied in
their own right because of the tremendous insight into galaxy formation they
provide. 

The purpose of this \tnote{} is to report on the (ongoing) work we have
undertaken to carry out a custom analysis of all the large galaxies in the
approximately $14,000$~deg$^{2}$ DESI footprint.  Focusing first on the DECaLS
Data Release~2 (DECaLS/DR2), we select an angular diameter limited sample of
$10,654$ galaxies with existing $grz$ imaging and we assess how the current
version of the pipeline handles sky-subtraction, deblending, and the photometry
of one test case, UGC04203.  We present the results of preliminary code written
to deal with some of the unique challenges posed by large galaxies (see
Section~\ref{sec:challenges}), and we conclude with a discussion of key goals
moving forward.

%In particular, large galaxies pose enough unique challenges (see
%Section~\ref{sec:challenges}) that they must be treated as a special case within
%the imaging datasets.

All the code written as part of this analysis is publicly accessible within the
{\tt legacypipe} Github repository\footnote{The code is currently in the {\tt
    largegalaxies} branch although it will be merged eventually into the {\tt
    master} branch.}, while this \tnote{} itself can be found in the DESI {\sc
  svn} repository at {\tt
  \url{https://desi.lbl.gov/svn/docs/technotes/imaging/large-galaxies}}.

\subsection{Challenges Posed by Large Galaxies}\label{sec:challenges}

Detecting, deblending, and modeling the surface brightness profiles of large
galaxies poses several key challenges for ground-based optical imaging surveys.
Here, we highlight some of these specific issues, in no particular order of
importance:
\begin{itemize}
%\item{{\em Flat-fielding}---Because large galaxies may subtend a significant
%  fraction of the field-of-view of the detector, excellent flat-fielding (if
%  possible, with sub-percent accuracy) is crucial.}
\item{{\em Sky subtraction}---Galaxies do not have sharp, or truncated edges,
  making it difficult to ascertain where the galaxy ends and the sky begins.
  Most standard sky-subtraction algorithms, which work perfectly well for stars
  and galaxies with typical (apparent) sizes, tend to subtract the light
  contained in the outer parts of galaxies.  Perhaps surprisingly, these outer,
  low surface-brightness isophotes may contain $50\%$ or more of the integrated
  light of the galaxy, depending on its morphological type, mass-assembly
  history, and large-scale environment.}
  %Consequently, large galaxies require customized and carefully constrained sky
  %subtraction. 
\item{{\em Morphological complexity}---Almost by definition, large galaxies are
  much better resolved spatially than their smaller (typically more distant)
  counterparts.  For example, visual inspection of large galaxies may reveal
  distinct bulge and disk components, bars and rings, spiral arms, dust lanes,
  tidally stripped material, azimuthal asymmetries, and other unique
  morphological characteristics (which is largely what makes large galaxies so
  fascinating to study!).  However, it is clear that single-component,
  azimuthally symmetric models (e.g., exponential or single-S\'{e}rsic) cannot
  fully capture this complexity, requiring more sophisticated and flexible
  galaxy models.  This need is especially acute when dealing with multi-band
  data, since many galaxy types exhibit radial (occasionally, non-monotonic!)
  color gradients.}
%\item{{\em Mergers} -- ??}
\item{{\em Detecting and deblending}---Because of their size, one of the most
  commonly encountered issues when analyzing large galaxies is \emph{shredding},
  which is when a single galaxy is (improperly) shredded into multiple disjoint
  components.  A related issue is when individual (resolved) components
  \emph{within} the galaxy (e.g., star-forming regions) are identified as
  distinct objects.  Similarly, even when the galaxy is not shredded, it is
  still challenging to efficiently detect and measure the colors of galaxies and
  stars located on top of or near large galaxies, owing to the non-uniform
  ``background'' these sources find themselves on.}
%\item{{\em Varying outer surface-brightness profiles} -- Intimately related to
%  the previous issue, the outer surface-brightness profiles of galaxies vary
%  widely, depending on the morphological type, mass-assembly history, and
%  environment of each galaxy.  In other words, the outer light-profiles of
%  galaxies cannot necessarily be extrapolated from the inner light-profiles.}
\end{itemize}

\noindent In this \tnote{} we begin to investigate these and other issues, with
the ultimate goal of implementing solutions to all of them.

