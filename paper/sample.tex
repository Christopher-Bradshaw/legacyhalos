\section{Sample Selection}\label{sec:sample}

Our goal is to define a parent sample of large galaxies---galaxies with large
angular diameters--- in Legacy Survey imaging.
%which will require special care
%during pre-processing (e.g., when the sky is subtracted) and when they are
%processed by the {\tt legacypipe} photometric pipeline.  
Although there are many different published catalogs of large galaxies, we have
found the {\tt HyperLeda}\footnote{\url{http://leda.univ-lyon1.fr}}
extragalactic database to be the most comprehensive and homogeneous.  We select
our parent sample by executing the following two SQL queries:\footnote{See
  \url{http://leda.univ-lyon1.fr/leda/fullsql.html} for the SQL interface and
  \url{http://leda.univ-lyon1.fr/leda/ldoc.html} for documentation on each of
  the stored quantities.  Note that two separate queries are necessary to stay
  within the time and memory limits of the SQL server.  }

\begin{verbatim}
SELECT
  pgc, objname, objtype, al2000, de2000, type, 
  multiple, logd25, logr25, pa, bt, it, v
WHERE 
  logd25 > 0.05 AND logd25 < 0.5 and (objtype='G' or objtype='g' or 
  objtype='M' or objtype='M2' or objtype='M3' or objtype='MG' or 
  objtype='MC')
\end{verbatim}

\noindent and

\begin{verbatim}
SELECT
  pgc, objname, objtype, al2000, de2000, type, 
  multiple, logd25, logr25, pa, bt, it, v
WHERE 
  logd25 > 0.5 and (objtype='G' or objtype='g' or 
  objtype='M' or objtype='M2' or objtype='M3' or objtype='MG' or 
  objtype='MC')
\end{verbatim}

%\begin{figure}[!ht]
%\centering
%\includegraphics[width=0.9\textwidth]{figures/lslga-dr2-radec.pdf}
%\caption{Celestial positions of the galaxies in our parent sample (blue
%  colormap) and our final angular-diameter limited sample of $\approx10,000$
%  galaxies with DECaLS/DR2 $grz$ imaging (red squares).  We render the parent
%  sample of $>2$~million galaxies using a logarithmic density
%  map. \label{fig:radec}}
%\end{figure}

\noindent The resulting two output tables, consisting of $2,143,628$ unique
galaxies (or unresolved galaxy pairs and triplets, e.g., mergers), are parsed
and combined into a single FITS catalog which is angular-diameter limited to
$0.11$~arcmin, or approximately $6.7$~arcsec (subject to the surface brightness
completeness of {\tt HyperLeda}).  Following historical precedent, the angular
diameter is given by $D(25)$, the diameter of the galaxy at the
$25$~mag~arcsec$^{-2}$ isophote in the $B$-band (Johnson, Vega).

Next, we reduce this sample further by applying two additional cuts.  First, we
restrict the sample to have $0.5<D(25)<10$~arcmin.  The lower limit was somewhat
arbitrarily chosen to reduce the sample to a more manageable size, while the
upper cut was applied to remove galaxies like Andromeda=M31 from the sample,
which require even more care and effort to analyze properly.  We note here that
$10$~arcmin is roughly the diameter of a single DECam CCD.  And finally, we
remove galaxies which do not have $grz$ imaging in DECaLS/DR2, leaving a final
sample of $10,654$ galaxies.

In Figure~\ref{fig:radec} we plot the positions of all the galaxies in the
parent sample (blue density map) and the galaxies in our final sample (red
squares).  Obviously, the size of the sample with three-band imaging will grow
significantly in subsequent DECaLS data releases, as well as once we begin to
incorporate data from the BASS and MzLS imaging surveys.

%\begin{figure}[!ht]
%\centering
%\includegraphics[width=0.9\textwidth]{figures/lslga-dr2-d25-bmag.pdf}
%\caption{Angular diameter, $D(25)$, versus $B$-band magnitude for the parent
%  sample (blue density map) and our final sample of galaxies with $grz$
%  DECaLS/DR2 imaging (red points).  The horizontal dashed lines bounded by 
%  $0.5<D(25)<10$~arcmin show the additional angular diameter cuts we apply to
%  select our final sample. \label{fig:d25}}
%\end{figure}

Finally, in Figure~\ref{fig:d25} we plot $D(25)$ versus $B$-band
magnitude\footnote{The $B$-band magnitudes are tabulated in {\tt HyperLeda} and
  are only meant to be indicative of the galaxy brightness.  Many of these
  magnitudes are based on decades-old photographic plates.} for both samples.
On this figure we annotate two particularly large galaxies---M31 and the Small
Magellanic Cloud---which are excluded by our angular diameter cuts (horizontal
dashed lines).  We plot our final sample of galaxies using small red points.

