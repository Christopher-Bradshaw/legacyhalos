\documentclass[preprint]{aastex61}
\usepackage{microtype}
\usepackage{url}
\usepackage{amsmath}
\usepackage{amssymb}
\usepackage{natbib}
\usepackage{multirow}

\pdfoutput=1
\bibliographystyle{aasjournal}

%%% This file is generated by the Makefile.
\newcommand{\githash}{6e5f9fc}\newcommand{\gitdate}{2017-07-28}\newcommand{\gitauthor}{John Moustakas}

%%%%%%%%%%%%%%%%%%%%%%%%%%%%%%%%%%%%%%%%%%%%%%%%%%
% Useful aliases
\newcommand{\project}[1]{\textsf{#1}}
\newcommand{\TODO}[1]{{\it \color{red} (#1)}}

%%%%%%%%%%%%%%%%%%%%%%%%%%%%%%%%%%%%%%%%%%%%%%%%%%
%\slugcomment{Submitted to the Astronomical Journal (AJ)}
\shorttitle{Legacy Survey Large Galaxy Atlas}
\shortauthors{Moustakas, Lang, et al.}

\begin{document}

\title{Legacy Survey Large Galaxy Atlas}

\author{John Moustakas}
\affil{Department of Physics \& Astronomy, Siena College, 515 Loudon Road,
  Loudonville, 12211}
\email{jmoustakas@siena.edu}

\author{Dustin Lang} \affil{Dunlap Institute for Astronomy \& Astrophysics and
  Department of Astronomy \& Astrophysics University of Toronto}
\email{dstndstn@gmail.com}

\begin{abstract}
I like cheese.
\end{abstract}

\section{Introduction}\label{sec:intro}

The Dark Energy Spectroscopic Instrument (DESI) will select spectroscopic
targets using data from three precursor ground-based optical imaging
surveys---DECaLS (DECam Legacy Survey), MzLS (Mayall $z$-band Legacy Survey),
and BASS (Beijing-Arizona Sky Survey).\footnote{\url{legacysurvey.org}} The
positions, shapes, sizes, colors, and other observable properties of the
galaxies and stars in these datasets are being measured by the {\tt
  legacypipe}\footnote{\url{https://github.com/legacysurvey/legacypipe}}
photometric pipeline, which uses the {\tt
  Tractor}\footnote{\url{http://thetractor.org}} to build a probabilistically
justified model of each source.

%and in particular target selection for the 

However, large galaxies---defined here to be galaxies with large (angular) sizes
projected on the sky, typically larger than $5-10$~arcsec---are not being
modeled properly by the production version of the pipeline, which will have
significant implications for DESI target selection, especially the Bright Galaxy
Survey (BGS).  A closely related issue is that each DESI pointing of 5000 fibers
will contain at least one galaxy $30$~arcsec or larger, so by analyzing large
galaxies we will be able to identify and select the desired fiber positions in
fields containing large galaxies (e.g., along the major axis or on the most
prominent star-forming regions).  Finally, large galaxies should be studied in
their own right because of the tremendous insight into galaxy formation they
provide. 

The purpose of this \tnote{} is to report on the (ongoing) work we have
undertaken to carry out a custom analysis of all the large galaxies in the
approximately $14,000$~deg$^{2}$ DESI footprint.  Focusing first on the DECaLS
Data Release~2 (DECaLS/DR2), we select an angular diameter limited sample of
$10,654$ galaxies with existing $grz$ imaging and we assess how the current
version of the pipeline handles sky-subtraction, deblending, and the photometry
of one test case, UGC04203.  We present the results of preliminary code written
to deal with some of the unique challenges posed by large galaxies (see
Section~\ref{sec:challenges}), and we conclude with a discussion of key goals
moving forward.

%In particular, large galaxies pose enough unique challenges (see
%Section~\ref{sec:challenges}) that they must be treated as a special case within
%the imaging datasets.

All the code written as part of this analysis is publicly accessible within the
{\tt legacypipe} Github repository\footnote{The code is currently in the {\tt
    largegalaxies} branch although it will be merged eventually into the {\tt
    master} branch.}, while this \tnote{} itself can be found in the DESI {\sc
  svn} repository at {\tt
  \url{https://desi.lbl.gov/svn/docs/technotes/imaging/large-galaxies}}.

\subsection{Challenges Posed by Large Galaxies}\label{sec:challenges}

Detecting, deblending, and modeling the surface brightness profiles of large
galaxies poses several key challenges for ground-based optical imaging surveys.
Here, we highlight some of these specific issues, in no particular order of
importance:
\begin{itemize*}
%\item{{\em Flat-fielding}---Because large galaxies may subtend a significant
%  fraction of the field-of-view of the detector, excellent flat-fielding (if
%  possible, with sub-percent accuracy) is crucial.}
\item{{\em Sky subtraction}---Galaxies do not have sharp, or truncated edges,
  making it difficult to ascertain where the galaxy ends and the sky begins.
  Most standard sky-subtraction algorithms, which work perfectly well for stars
  and galaxies with typical (apparent) sizes, tend to subtract the light
  contained in the outer parts of galaxies.  Perhaps surprisingly, these outer,
  low surface-brightness isophotes may contain $50\%$ or more of the integrated
  light of the galaxy, depending on its morphological type, mass-assembly
  history, and large-scale environment.}
  %Consequently, large galaxies require customized and carefully constrained sky
  %subtraction. 
\item{{\em Morphological complexity}---Almost by definition, large galaxies are
  much better resolved spatially than their smaller (typically more distant)
  counterparts.  For example, visual inspection of large galaxies may reveal
  distinct bulge and disk components, bars and rings, spiral arms, dust lanes,
  tidally stripped material, azimuthal asymmetries, and other unique
  morphological characteristics (which is largely what makes large galaxies so
  fascinating to study!).  However, it is clear that single-component,
  azimuthally symmetric models (e.g., exponential or single-S\'{e}rsic) cannot
  fully capture this complexity, requiring more sophisticated and flexible
  galaxy models.  This need is especially acute when dealing with multi-band
  data, since many galaxy types exhibit radial (occasionally, non-monotonic!)
  color gradients.}
%\item{{\em Mergers} -- ??}
\item{{\em Detecting and deblending}---Because of their size, one of the most
  commonly encountered issues when analyzing large galaxies is \emph{shredding},
  which is when a single galaxy is (improperly) shredded into multiple disjoint
  components.  A related issue is when individual (resolved) components
  \emph{within} the galaxy (e.g., star-forming regions) are identified as
  distinct objects.  Similarly, even when the galaxy is not shredded, it is
  still challenging to efficiently detect and measure the colors of galaxies and
  stars located on top of or near large galaxies, owing to the non-uniform
  ``background'' these sources find themselves on.}
%\item{{\em Varying outer surface-brightness profiles} -- Intimately related to
%  the previous issue, the outer surface-brightness profiles of galaxies vary
%  widely, depending on the morphological type, mass-assembly history, and
%  environment of each galaxy.  In other words, the outer light-profiles of
%  galaxies cannot necessarily be extrapolated from the inner light-profiles.}
\end{itemize*}

\noindent In this \tnote{} we begin to investigate these and other issues, with
the ultimate goal of implementing solutions to all of them.



\section{Sample Selection}\label{sec:sample}

Our goal is to define a parent sample of large galaxies---galaxies with large
angular diameters--- in Legacy Survey imaging.
%which will require special care
%during pre-processing (e.g., when the sky is subtracted) and when they are
%processed by the {\tt legacypipe} photometric pipeline.  
Although there are many different published catalogs of large galaxies, we have
found the {\tt HyperLeda}\footnote{\url{http://leda.univ-lyon1.fr}}
extragalactic database to be the most comprehensive and homogeneous.  We select
our parent sample by executing the following two SQL queries:\footnote{See
  \url{http://leda.univ-lyon1.fr/leda/fullsql.html} for the SQL interface and
  \url{http://leda.univ-lyon1.fr/leda/ldoc.html} for documentation on each of
  the stored quantities.  Note that two separate queries are necessary to stay
  within the time and memory limits of the SQL server.  }

\begin{verbatim}
SELECT
  pgc, objname, objtype, al2000, de2000, type, 
  multiple, logd25, logr25, pa, bt, it, v
WHERE 
  logd25 > 0.05 AND logd25 < 0.5 and (objtype='G' or objtype='g' or 
  objtype='M' or objtype='M2' or objtype='M3' or objtype='MG' or 
  objtype='MC')
\end{verbatim}

\noindent and

\begin{verbatim}
SELECT
  pgc, objname, objtype, al2000, de2000, type, 
  multiple, logd25, logr25, pa, bt, it, v
WHERE 
  logd25 > 0.5 and (objtype='G' or objtype='g' or 
  objtype='M' or objtype='M2' or objtype='M3' or objtype='MG' or 
  objtype='MC')
\end{verbatim}

%\begin{figure}[!ht]
%\centering
%\includegraphics[width=0.9\textwidth]{figures/lslga-dr2-radec.pdf}
%\caption{Celestial positions of the galaxies in our parent sample (blue
%  colormap) and our final angular-diameter limited sample of $\approx10,000$
%  galaxies with DECaLS/DR2 $grz$ imaging (red squares).  We render the parent
%  sample of $>2$~million galaxies using a logarithmic density
%  map. \label{fig:radec}}
%\end{figure}

\noindent The resulting two output tables, consisting of $2,143,628$ unique
galaxies (or unresolved galaxy pairs and triplets, e.g., mergers), are parsed
and combined into a single FITS catalog which is angular-diameter limited to
$0.11$~arcmin, or approximately $6.7$~arcsec (subject to the surface brightness
completeness of {\tt HyperLeda}).  Following historical precedent, the angular
diameter is given by $D(25)$, the diameter of the galaxy at the
$25$~mag~arcsec$^{-2}$ isophote in the $B$-band (Johnson, Vega).

Next, we reduce this sample further by applying two additional cuts.  First, we
restrict the sample to have $0.5<D(25)<10$~arcmin.  The lower limit was somewhat
arbitrarily chosen to reduce the sample to a more manageable size, while the
upper cut was applied to remove galaxies like Andromeda=M31 from the sample,
which require even more care and effort to analyze properly.  We note here that
$10$~arcmin is roughly the diameter of a single DECam CCD.  And finally, we
remove galaxies which do not have $grz$ imaging in DECaLS/DR2, leaving a final
sample of $10,654$ galaxies.

In Figure~\ref{fig:radec} we plot the positions of all the galaxies in the
parent sample (blue density map) and the galaxies in our final sample (red
squares).  Obviously, the size of the sample with three-band imaging will grow
significantly in subsequent DECaLS data releases, as well as once we begin to
incorporate data from the BASS and MzLS imaging surveys.

%\begin{figure}[!ht]
%\centering
%\includegraphics[width=0.9\textwidth]{figures/lslga-dr2-d25-bmag.pdf}
%\caption{Angular diameter, $D(25)$, versus $B$-band magnitude for the parent
%  sample (blue density map) and our final sample of galaxies with $grz$
%  DECaLS/DR2 imaging (red points).  The horizontal dashed lines bounded by 
%  $0.5<D(25)<10$~arcmin show the additional angular diameter cuts we apply to
%  select our final sample. \label{fig:d25}}
%\end{figure}

Finally, in Figure~\ref{fig:d25} we plot $D(25)$ versus $B$-band
magnitude\footnote{The $B$-band magnitudes are tabulated in {\tt HyperLeda} and
  are only meant to be indicative of the galaxy brightness.  Many of these
  magnitudes are based on decades-old photographic plates.} for both samples.
On this figure we annotate two particularly large galaxies---M31 and the Small
Magellanic Cloud---which are excluded by our angular diameter cuts (horizontal
dashed lines).  We plot our final sample of galaxies using small red points.



\section{Modeling Large Galaxies}\label{sec:modeling}

To highlight our progress modeling the sample of large galaxies, we focus on one
test case, UGC04203 (also known as the Phoenix galaxy), a face-on Sa galaxy with
an angular diameter of $0.85$~arcmin.  In Section~\ref{sec:production} we
summarize the performance of the production (DR2) version of the {\tt
  legacypipe} pipeline, in Section~\ref{sec:sky} we discuss masking and
sky-subtraction around large galaxies, and in Section~\ref{sec:largepipeline} we
show preliminary results from a version of the pipeline which is customized for
large galaxies.

\begin{figure}%[!ht]
\centering
\includegraphics[width=0.3\textwidth]{figures/ugc04203-image-runbrick-annot.jpg}
\includegraphics[width=0.3\textwidth]{figures/ugc04203-model-runbrick-annot.jpg}
\includegraphics[width=0.3\textwidth]{figures/ugc04203-resid-runbrick-annot.jpg}
\caption{Modeling of UGC04203 using the production (DR2) version of the
  {\tt legacypipe} pipeline. 
%  See Section~\ref{sec:production} for details.
  \label{fig:runbrick}} 
\end{figure}

\subsection{Production Pipeline}\label{sec:production}

In Figure~\ref{fig:runbrick} we illustrate the performance of the DR2 version of
the {\tt legacypipe} pipeline.  The left, middle, and right panels show the
data, model, and residual (data minus model) color mosaics.  The small circles
in each panel identify detected sources, where the color of the circle
corresponds to the final morphological classification of the object: {\tt
  PSF}=white, {\tt SIMP}=red, {\tt EXP}=orange, {\tt DEV}=cyan, and {\tt
  COMP}=yellow.\footnote{See http://legacysurvey.org/dr2/catalogs for a
  description of these classifications.}  Sources are identified as $5 \sigma$
peaks above the rms noise of the image after convolving with a Gaussian kernel
whose full-width at half-maximum (FWHM) is given by the median seeing of the
data. 

Overall, we find that the DR2 pipeline is doing a terrible job of modeling the
large, central galaxy.  Although UGC04203 has been successfully classified as a
{\tt COMP} (linear combination of an exponential plus de~Vaucouleurs
surface-brightness profile), its center has been significantly over-subtracted
and the outer envelope has been under-subtracted.  Moreover, the central panel
(model) shows clear aliasing in Fourier space.  Finally, we find that several
significant sources have not been identified, including two objects in the outer
envelope of UGC04203 (lower-right quadrant) and one source in the bottom-left
part of the footprint.  Failing to detect these sources creates several
significant issues when the pipeline attempts to optimize the parameters
(shapes, sizes, and surface-brightness profiles) of the sources it \emph{does}
detect.

\subsection{Improved Masking and Sky Subtraction}\label{sec:sky}

Unlike the typical, ``small'' stars and galaxies which appear in Legacy Survey
imaging, large galaxies have extended outer envelopes which may include a
significant amount of light.  Consequently, both object masking and
sky-subtraction are critically important.  Masking and sky-subtraction, of
course, must occur at the CCD-level data, since the {\tt Tractor} operates from
the unremapped, unstacked, pixel-level data.

To illustrate the available multi-band imaging for UGC04203, in
Figure~\ref{fig:ccdpos} we plot the positions of the $g$-band (left panel),
$r$-band (middle panel), and $z$-band (right panel) exposures, where each
individual CCD has been color-coded as indicated in the legend.  For reference,
the size of the dashed square is five times the angular diameter of the galaxy,
and the footprint covered by the square corresponds to the sky region shown in
Figure~\ref{fig:runbrick}.  Finally, the small circle at the center of each
panel in Figure~\ref{fig:ccdpos} indicates the angular diameter of UGC04203.

\begin{figure}%[!ht]
\centering
\includegraphics[width=0.9\textwidth]{figures/qa-ugc04203-ccdpos.png}
\caption{Positions of the CCDs for the available (left to right) $g$-, $r$-, and
  $z$-band imaging of UGC04203.  The dashed square is five times the angular
  diameter of the galaxy and the small circle at the center of each panel
  indicates the angular diameter of UGC04203.  \label{fig:ccdpos}}
\end{figure}

In Figure~\ref{fig:qaccd} we focus on one of these CCDs---{\tt ccd04}, an
$r$-band exposure with UGC04203 positioned reasonably close to the center of the
field---in more detail.  In this figure we show the image (including the
sky), the object mask constructed by the DR2 pipeline (one and zero indicate
masked and unmasked pixels, respectively), the object mask generated by the
large-galaxy pipeline, the DR2 model of the sky, and finally the model of the
sky based on the customized large-galaxy pipeline.  The white circle in each
panel shows the position and angular extent of UGC04203.

Figure~\ref{fig:qaccd} shows two key ideas.  First, the large-galaxy pipeline is
much more aggressive at masking pixels containing astrophysical sources,
especially the outer isophotes of UGC04203.  And second, the production version
of the pipeline clearly over-subtracts the sky in and around UGC04203, whereas
the large-galaxy pipeline (currently) subtracts a uniform sky background from
the data.  Other diagnostic plots (not shown here) indicate that the
low-resolution spline sky-subtraction implemented in the DR2 version of {\tt
  legacypipe} is overly aggressive (flexible) around large galaxies, and can
lead to the kind of over-subtraction systematics shown in Figure~\ref{fig:qaccd}
(fourth panel from the left).

%The units of the images are nanomaggies.

%\begin{figure}%[!ht]
%\centering
%\includegraphics[width=0.5\textwidth]{figures/qa-ugc04203-blobs.png}
%\caption{Blobs.  \label{fig:blobs}}
%\end{figure}

\begin{figure}[!ht]
\centering
\includegraphics[width=1.0\textwidth]{figures/qa-ugc04203-ccd04-2d.png}
\caption{CCD-level diagnostic plots for a single $r$-band exposure of UGC04203
  (marked with a cirle).  From left to right we show the image, the object mask
  constructed by the DR2 pipeline, the object mask generated by the large-galaxy
  pipeline, the DR2 model of the sky, and finally the model of the sky based on
  the customized large-galaxy pipeline. \label{fig:qaccd}}
\end{figure}

\subsection{Large-Galaxy Pipeline}\label{sec:largepipeline}

Figure~\ref{fig:custom} shows the final result of running the customized
large-galaxy pipeline on UGC04203.  As in Figure~\ref{fig:runbrick} (see also
Section~\ref{sec:production}), we show---from left to right---a color montage of
the data, the model, and the residuals, and we show the detected sources and
their classification using colored circles.  Overall, we find a significant
improvement in the two-dimensional model of UGC04203, which as before is
classified as a composite galaxy.  In particular, the residual image shows
astrophysically interesting features, including spiral structure and low-level
shells at large galactocentric radius, the latter of which is generally cited as
evidence of a merger-driven origin for spheroidal and bulge-dominated galaxies. 

One outstanding issue shown in Figure~\ref{fig:custom}, however, is the source
in the lower-left quadrant which fails to be detected.  To compensate for this
extraneous flux, the pipeline chooses a thin, almost needle-shaped galaxy
profile for the adjacent source.  Although this issue does not specifically
affect our model of UGC04203, it is a recurring issue which we hope to address
in a future version of the {\tt legacypipe} pipeline.



\section{Discussion}\label{sec:discussion}
Discuss. 


\begin{itemize}

\item{Look at van Dokkum, Conroy et al. (2017) -- the central regions of massive
  galaxies really seem to be special, with the IMF slope varying as a function
  of velocity dispersion $\sigma$.  Connect this with the two-phase formation
  paradigm for massive galaxies, where the central regions form first and then
  grow inside-out.  What are the implications here on our SED modeling for the
  integrated stellar mass function?  For example, consider a bursty star
  formation history, bottom heavy IMF for the central region and then a more
  quiescent, Salpeter-like IMF for the outer regions.}

\item{Think about how AGB stars will affect the WISE photometry.}

\end{itemize}


\section{Conclusions \& Next Steps}\label{sec:summary}

This \tnote{} reports on our ongoing effort to develop a modified version of the
{\tt legacypipe} photometric pipeline which is optimized for large galaxies,
with the ultimate goal of producing a Legacy Survey Large Galaxy Atlas (LSLGA).
In addition to enabling detailing multi-wavelength analyses of large, spatially
well-resolved galaxies, this effort will also significantly improve DESI target
selection for the Bright Galaxy Survey and in pointings containing large
galaxies.

In addition to addressing the outstanding issues discussed above, in the near
future we intend to work on the following additional threads, in no particular
order of importance:
\begin{itemize*}
\item{We will revisit the parent galaxy sample selection to ensure that low
  surface brightness galaxies are not being excluded (e.g., by incorporating
  H~{\sc i}-selected galaxies).}
\item{We will run the large-galaxy pipeline on the full sample of $10,654$
  galaxies, in order to identify the most common problems and failure modes.}
\item{We will build a web-based interface on the {\tt legacysurvey.org}
  web-server (work that we have already begun), in order to make it easy to
  inspect the data and model outputs for the large-galaxy sample.}
\item{Using existing code, we will use image simulations---whereby we insert
  simulated large galaxies into the CCD-level data---to test the performance of
  the large-galaxy pipeline under different initial conditions and in different
  regimes.}
\item{We will incorporate more sophisticated galaxy models, including
  multi-component S\'{e}rsic profiles, into the library of possible
  two-dimensional galaxy models.}
\item{We will investigate ways of accounting for second-order variations in the
  galaxy models, such as radial color gradients or azimuthal asymmetries.}
\item{And finally we will engage with the Data Systems and Target Selection
  Working Group to incorporate the catalogs constructed using the large-galaxy
  pipeline into DESI target selection and fiber assignment.}
\end{itemize*}

%Planar sky.

%We zoom in on the object of interest and define a custom ``brick'' centered on
%that galaxy -- no edge issues to worry about!

\begin{figure}
\centering
\includegraphics[width=0.3\textwidth]{figures/ugc04203-image-custom-annot.jpg}
\includegraphics[width=0.3\textwidth]{figures/ugc04203-model-custom-annot.jpg}
\includegraphics[width=0.3\textwidth]{figures/ugc04203-resid-custom-annot.jpg}
\caption{Modeling of UGC04203 using the customized, large-galaxy optimized
  version of the pipeline.  See Section~\ref{sec:largepipeline} for details.
\label{fig:custom}}  
\end{figure}


%\bibliographystyle{apj}
%\bibliography{/Users/ioannis/bibdesk/ioannis}
%\input{target-truth.bbl}

\end{document}


